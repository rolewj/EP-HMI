\documentclass{article}
\usepackage[utf8x]{inputenc}
\usepackage[T2A]{fontenc}
\usepackage[russian]{babel}

\usepackage{xcolor}
\usepackage{listings}
\usepackage{caption}
\usepackage[left=3cm,right=1.5cm,top=1cm,bottom=2cm]{geometry}

\definecolor{backgroundColour}{rgb}{0.95,0.95,0.92}

\DeclareCaptionFont{white}{\color{white}}
\DeclareCaptionFormat{listing}{\colorbox{gray}{\parbox{\dimexpr\textwidth-2\fboxsep\relax}{#1#2#3}}}
\captionsetup[lstlisting]{format=listing,labelfont=white,textfont=white}

\lstset{
language=C++,
backgroundcolor=\color{backgroundColour},
title=\lstname,
basicstyle=\ttfamily\footnotesize,
keywordstyle=\color{blue},
stringstyle=\color{red},
commentstyle=\color{green},
numberstyle=\tiny,
numbers=left,
stepnumber=1,
numbersep=5pt,
showspaces=false,
showstringspaces=false,
showtabs=false,
frame=single,
tabsize=4,
captionpos=t,
breaklines=true,
breakatwhitespace=false
}


\begin{document}


\title{Стиль кода}
\author{Автор: Ющенко Андрей Викторович, ИА-032,\\ email: yuschenko2002@gmail.com,\\ github: @rolewj}
\date{Февраль 2022}

\maketitle

\section{C}
Используется версия С17 \cite{C}.\vspace{5mm}

\textbf{Инициализация переменных.} Ставятся пробелы до и после знака присваивания <<=>>.
\begin{lstlisting}[caption=Инициализация переменных.]
int x = 1;
int y = 1;
int b;
\end{lstlisting}

\textbf{Ввод и вывод данных.} Пробел ставится после запятой.
\begin{lstlisting}[caption=Ввод и вывод данных.]
scanf("%d", &n);
printf("Fibonacci number: %d", y);
\end{lstlisting}

\textbf{Условный оператор if.} Ставятся пробелы после <<if>> и <<else>>, до и после знаков опираций, а также до фигурной скобки. В виде отступов используется табуляция (4 пробела).
\begin{lstlisting}[caption=Условный оператор if.]
if (b > 5) {
    b = b + 5;
}
else {
    b = b - 5;
}
\end{lstlisting}

\textbf{Цикл for.} Использование стиля аналогично условному опреатору if.
\begin{lstlisting}[caption=Цикл for.]
for (int i = 2; i < n; i++) {
    y = x + y;
    x = y - x;
}
\end{lstlisting}

\textbf{Реализация функции}, например, fibonacci.
\begin{lstlisting}[caption=Реализация функции.]
int fibonacci(int n) {
    if (n == 1 || n == 2) {
        return 1;
    } 
    else {
        return fibonacci(n - 1) + fibonacci(n - 2);
    }
}
\end{lstlisting}

\section{C++}
Используется версия С++20 \cite{C++}.\vspace{5mm}

Применяются стили написания, аналогично языку C. Поэтому следует описать ввод и вывод данных, а также классы, которые не используются в языке C.\par
\textbf{Ввод и вывод данных.} Ставятся пробелы до и после операторов ввода и вывода.
\begin{lstlisting}[caption=Ввод и вывод данных.]
std::cout << "Enter a number: ";
std::cin >> a;
std::cout << "You entered " << a << std::endl;
\end{lstlisting}

\textbf{Классы.} Пробел ставится до фигурной скобки, табуляция~--- перед public/private и после них со следующей строки.
\begin{lstlisting}[caption=Классы.]
class Students {
    public:
        void set_name(std::string);
        std::string get_name();
        void set_last_name(std::string);
        std::string get_last_name();
        void set_scores(int []);
        void set_average_grade(float);
        float get_average_grade();

    private:
        int scores[5];
        float average_grade;
        std::string name;
        std::string last_name;
};
\end{lstlisting}

\begin{thebibliography}{}
\bibitem{C}
ISO/IEC 9899 Programming languages~--- C.
\bibitem{C++}
ISO/IEC 14882 Programming languages~--- C++.
\end{thebibliography}

\end{document}